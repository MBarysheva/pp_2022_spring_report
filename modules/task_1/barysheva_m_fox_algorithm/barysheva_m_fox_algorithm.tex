
\documentclass[12pt]{article}
\usepackage{lingmacros}
\usepackage{tree-dvips}

\usepackage[T2A]{fontenc}
\usepackage[utf8]{luainputenc}
\usepackage[english, russian]{babel}
\usepackage[pdftex]{hyperref}
\usepackage[14pt]{extsizes}
\usepackage{listings}
\usepackage{amsmath}
\usepackage{color}
\usepackage{multicol}
\usepackage{longtable}
\usepackage{geometry}
\usepackage{enumitem}
\usepackage{multirow}
\usepackage{graphicx}
\usepackage{indentfirst}
\usepackage{caption}

\geometry{a4paper,top=2cm,bottom=2cm,left=2.5cm,right=1.5cm}
\setlength{\parskip}{0.5cm}

\lstset{language=C++,
		basicstyle=\footnotesize,
		keywordstyle=\color{blue}\ttfamily,
		stringstyle=\color{red}\ttfamily,
		commentstyle=\color{green}\ttfamily,
		morecomment=[l][\color{magenta}]{\#}, 
		tabsize=4,
		breaklines=true,
  		breakatwhitespace=true,
  		title=\lstname,       
}
\renewcommand{\thesubsection}{\arabic{subsection}}
\makeatletter
\def\@seccntformat#1{\@ifundefined{#1@cntformat}%
   {\csname the#1\endcsname\quad}
   {\csname #1@cntformat\endcsname}}
\newcommand\section@cntformat{}
\makeatother

\begin{document}

\begin{titlepage}

\begin{center}
Министерство науки и высшего образования Российской Федерации
\end{center}

\begin{center}
Федеральное государственное автономное образовательное учреждение высшего образования \\
Национальный исследовательский Нижегородский государственный университет им. Н.И. Лобачевского
\end{center}

\begin{center}
Институт информационных технологий, математики и механики
\end{center}

\vspace{4em}

\begin{center}
\textbf{\LargeОтчет по лабораторной работе} \\
\end{center}
\begin{center}
\textbf{\LargeУмножение плотных матриц. Элементы типа double. Блочная схема, алгоритм Фокса.} \\
\end{center}

\vspace{4em}

\newbox{\lbox}
\savebox{\lbox}{\hbox{text}}
\newlength{\maxl}
\setlength{\maxl}{\wd\lbox}
\hfill\parbox{7cm}{
\hspace*{5cm}\hspace*{-5cm}\textbf{Выполнила:} \\ студентка группы 381906-2 \\ Барышева М. А. \\
\\
\hspace*{5cm}\hspace*{-5cm}\textbf{Проверил:}\\ доцент кафедры МОСТ, \\ кандидат технических наук \\ Сысоев А. В.\\
}
\vspace{\fill}

\begin{center} Нижний Новгород \\ 2022 \end{center}

\end{titlepage}


\setcounter{page}{2}

% Содержание
\tableofcontents
\newpage

\section*{Введение}
\addcontentsline{toc}{section}{Введение}
Умножение матриц -- это одна из самых основных математических операций. Существует множество различных алгоритмов умножения матриц, блочные, линейные и т.д.. Прямое использование математического определения умножения матриц даёт алгоритм, занимающий \( O \left( n^{3} \right)  \) времени, что достаточно неэффективно. Разумеется, одни алгоритмы работают в разы быстрее других, но наиболее эффективным способом ускорения реализации умножения является алгоритм распараллеливания.\par

В рамках данной лабораторной работы мы реализуем алгоритм Фокса, используя различные технологии распараллеливания.\par

\newpage

\section*{Цель работы}
\addcontentsline{toc}{section}{Цель работы}
В рамках данной лабораторной работы необходимо разработать следующие компоненты:

\begin{itemize}
	\item Выполнить реализацию последовательного алгоритма умножения матриц;

	\item Выполнить реализацию алгоритма Фокса с использованием технологии OpenMP;

	\item Выполнить реализацию алгоритма Фокса с использованием технологии TBB;

	\item Выполнить реализацию алгоритма Фокса с использованием технологии std::Threads C++11;
	
	\item Провести расчет теоретического ускорения и эффективности;

\item Провести ряд тестов. Сравнить ускорение параллельного и не параллельного алгоритма;
\end{itemize}

\newpage

\section*{Описание алгоритма}
\addcontentsline{toc}{section}{Описание алгоритма}
Используется блочная схема разбиения матрицы. При таком способе разделения данных исходные матрицы A, B и результирующая матрица С представляются в виде наборов блоков. Далее предполагается что все матрицы являются квадратными размера $n*n$, количество блоков по горизонтали и вертикали одинаково и равно q (т.е. размер всех блоков равен $k*k$, $k=n/q$). При таком представлении данных операция матричного умножения, например, матрицы А в блочном виде может быть представлена так:\\
 \begin{equation}A_{00}= \left( \begin{matrix}
a_{00}  &  a_{01}\\
a_{10}  &  a_{11}\\
\end{matrix}
 \right) ,~~A_{01}= \left( \begin{matrix}
a_{02}  &   a_{03}\\
a_{12}  &   a_{13}\\
\end{matrix}
 \right) , A_{10}= \left( \begin{matrix}
a_{20}  &  a_{21}\\
a_{30}  &  a_{31}\\
\end{matrix}
 \right) ,~~A_{11}= \left( \begin{matrix}
a_{22}  &  a_{23}\\
a_{32}  &  a_{33}\\
\end{matrix}
 \right)  \end{equation} 
Каждый блок результирующей матрицы С определяется по формуле:
\begin{equation}
  C=A \times B=a_{i0}b_{0j}+a_{i1}b_{1j}+ \ldots +a_{i,n-1}b_{n-1,j}= \sum _{k=0}^{n-1}a_{ik}b_{kj} 
\end{equation}


\newpage

\section*{Методы распараллеливания}
\addcontentsline{toc}{section}{Методы распараллеливания}
Для начала создаются подзадачи вычисления отдельных блоков матрицы C,  при этом в подзадачах на каждой итерации расчетов располагается только по одному блоку исходных матриц A и B. Для нумерации подзадач будем использовать индексы размещаемых в подзадачах блоков матрицы C, т.е. подзадача $(i,j)$ отвечает за вычисление блока $C_{i,j}$ – тем самым, набор подзадач образует квадратную решетку, соответствующую структуре блочного представления матрицы C.  
\subsection{Реализация OpenMP}
\textit{Для распараллеливания используется директива $\#$ pragma omp parallel.}\\ \\
Каждому потоку присваивается некоторая характеристика – координаты в схеме. Они показывают, в какой блок результирующей матрицы будет записан ответ по завершении работы потока. На каждой итерации шагаем от 0 до $q$, где  $ q=\sqrt[]{num \_ threads}$, высчитывая нужный индекс по формуле $k\_bar$ = $(thread\_i+step)$ $\%$  $q$. Перемножаем соответствующие блоки в строке и столбце во временную результирующую матрицу tmp и суммируем с результирующей матрицей C. Эффективность высчитываем с помощью функции \textit{omp\_get\_wtime()}. 

\subsection{Реализация TBB}
Реализация в TBB основана на структуре \textit{blocked\_range2d} -- двумерного пространства для разбиения блоков и функции \textit{parallel\_for}, которая принимает lambda-функцию, перемножающую соответствующие блоки в строке и столбце во временную результирующую матрицу tmp и суммирующую с результирующей матрицей C. Эффективность высчитываем с помощью функции \textit{tbb::tick\_count}.

\subsection{Реализация std::threads}
В данной реализации мы создаём некоторое число потоков \textit{std::vector<std::thread> threads(q * q)}, которым при конструировании в качестве параметра передаём lambda-функцию, перемножающую соответствующие блоки в строке и столбце во временную результирующую матрицу tmp и суммирующую с результирующей матрицей C. Эффективность высчитываем с помощью функции \textit{CLOCKS\_PER\_SEC}.

\newpage

\section*{Тестирование}
\addcontentsline{toc}{section}{Тестирование}
Для проверки правильности алгоритмов была использована библиотека модульного тестирования Google C++ Testing Framework.

\newpage

\section*{Результаты экспериментов}
\addcontentsline{toc}{section}{Результаты экспериментов}
\begin{itemize}
	\item Операционная система: Windows 8.1
	\item Процессор: 
	Intel(R) Core(TM) i5-3210M CPU @ 2.50GHz\\
	Максимальная скорость:	2,50 ГГц\\
	Сокетов:	1\\
	Ядра:	2\\
	Логических процессоров:	4
	\item RAM: 8 Gb
\end{itemize}
\begin{center}
    

\begin{table}[!h]
\resizebox{\textwidth}{!}{
\begin{tabular}{|c|c|c|c|c|c|c|c|}
\hline
\multirow{2}{*}{Размер матрицы} & Последовательное выполнение & \multicolumn{2}{c|}{OpenMP} & \multicolumn{2}{c|}{TBB} & \multicolumn{2}{c|}{std::threads} \\ \cline{2-8} 
     & Время, сек & Время, сек & Ускорение & Время, сек & Ускорение & Время, сек & Ускорение \\ \hline
500  & 2.64       & 1.23       & 2.14      & 1.24       & 2.12      & 1.22       & 2.16      \\ \hline
1000 & 21.92      & 12.41      & 1.76      & 12.24      & 1.79      & 11.75      & 1.86      \\ \hline
\end{tabular}
}
\caption{Результаты экспериментов}
\end{table}
\end{center}


\newpage

\section*{Заключение}
\addcontentsline{toc}{section}{Заключение}
В результате данной лабораторной работы был изучен алгоритм Фокса параллельного умножения матриц, получен навык работы со стандартами для распараллеливания OpenMP, TBB, std::threads. Тесты работают, параллельные вычисления имеют существенное ускорение в сравнении с последовательными.\par

\newpage

\begin{thebibliography}{}
\addcontentsline{toc}{section}{Список литературы и ссылок}
\bibitem{Sysoev} Сысоев А.В., Мееров И.Б., Свистунов А.Н., Курылев А.Л., Сенин А.В., Шишков А.В., Корняков К.В., Сиднев А.А. «Параллельное программирование в системах с общей памятью. Инструментальная поддержка». Учебно-методические материалы по программе повышения квалификации «Технологии высокопроизводительных вычислений для обеспечения учебного процесса и научных исследований». Нижний Новгород, 2007, 110 с. 
\bibitem{Fox algorithm} Алгоритм Фокса перемножения матриц [Электронный ресурс] // URL: http://www.hpcc.unn.ru/?dir=1034


\end{thebibliography}

\newpage

\section*{Приложение}
\addcontentsline{toc}{section}{Приложение}
Код, написанный в рамках лабораторной работы представлен в следующем репозитории.
\begin{lstlisting}
https://github.com/MBarysheva/pp_2022_spring
\end{lstlisting}
\end{document}